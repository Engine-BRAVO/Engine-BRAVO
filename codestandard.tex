\documentclass{article}
\usepackage{fancyvrb}

\begin{document}

\section*{Coding Guidelines}

\begin{itemize}
    \item All code and comments are in English. Not American English. So these are right: \texttt{colour}, \texttt{centre}, \texttt{initialiser}. These are wrong: \texttt{color}, \texttt{center}, \texttt{initializer}.
    \item Every comma is followed by a space, so for example:
    \begin{Verbatim}[fontsize=\small]
    myFunction(5, 7, 8)
    \end{Verbatim}
    \item Spaces around operators, so for examanimatie + matrix berkeningen(rotere, schalen etc)ple \texttt{5 + 7} instead of \texttt{5+7}.
    \item Tabs have the same size as four spaces. Tabs are stored as tabs, not as spaces.
    \item Functions and static variables are in the same order in the \texttt{.h} and \texttt{.cpp} files.
    \item Use \texttt{\#pragma once} instead of include guards (we recently switched to this so you will still see a lot of old include guards in our code).
    \item Try to make short functions, preferably no more than 50 lines.
    \item Avoid making very large classes. Split a class whenever you anticipate the class will grow a lot and you can cleanly split it. In general, try to keep class sizes below 750 lines. However, don't split a class if the split would result in messy code. Some examples of messy divorces are: too tightly coupled classes, friend classes, and complex class hierarchies.
    \item Use forward declaration where possible: put as few includes in header-files as possible. Use \texttt{class Bert;} instead and move the \texttt{\#include "Bert.h"} to the \texttt{.cpp}-file.
    \item Includes and forward declarations are grouped as follows:
    \begin{itemize}
        \item Forward declarations from our own code (in alphabetical order)
        \item Includes from our own code (in alphabetical order)
        \item Empty line
        \item Per library:
        \begin{itemize}
            \item Forward declarations from that library (in alphabetical order)
            \item Includes from that library (in alphabetical order)
        \end{itemize}
    \end{itemize}
    \item The include of the \texttt{.h}-file belonging to this \texttt{.cpp} file is always at the top.
    \item Do not define static variables in a function. Use (static) class member variables instead.
    \item Everything must be const correct.
    \item Names of variables and functions should be descriptive. Long names are no problem, undescriptive names are. Only use abbreviations if they are very clear and commonly known.
    \item The first letter of classes, structs, and enum type names is a capital. Variables and functions start without a capital. Each further word in a name starts with a capital. Do not use underscores ( \texttt{\_} ) in variable names. So like this:
    \begin{Verbatim}[fontsize=\small]
    class MyClass
    {
        void someFunction();
        int someVariable;
    };
    \end{Verbatim}
    Member variables start with a lowercase \texttt{m} like \texttt{mMember} and arguments start with a lowercase \texttt{a} like \texttt{aArgument}.
    \item Implementation of functions is never in \texttt{.h}-files.
    \item Templatised functions that cannot be implemented in the \texttt{.cpp} file are implemented in an extra header file that is included from the main header file. Such a class can have 3 files: \texttt{MyClass.h}, \texttt{MyClassImplementation.h}, and \texttt{MyClass.cpp}. So like this:
    \begin{Verbatim}[fontsize=\small]
    class MyClass
    {
        template <typename T>
        void doStuff(T thingy);
    };

    #include "MyClassImplementation.h"
    \end{Verbatim}
    \item Start template typenames with \texttt{T}. If more info is relevant, you can add words after that, like \texttt{TString}.
    \item Several classes can never be defined in the same header file, except if a class is part of another class (i.e., defined inside the other class).
    \item Between functions, there are two empty lines.
    \item Use empty lines to structure and group code for readability.
    \item Add loads of comments to code.
    \item Write a short explanation of what a class does above each class. Especially make sure to explain relations there (e.g., “This class helps class X by doing Y for him”).
    \item Write comments above classes and functions with \texttt{///} instead of \texttt{//} (this way the Linux tools will also show info when hovering over code).
    \item Curly braces \texttt{\{} and \texttt{\}} always get their own line, so they are not put on the same line as the \texttt{if} or the \texttt{for}. The only exception is if you have lots of similar single-line \texttt{if}-statements right beneath each other. In that case, it is okay to put them on a single line. Like in this example:
    \begin{Verbatim}[fontsize=\small]
    if      ( banana &&  kiwi && length > 5) return cow;
    else if ( banana && !kiwi && length > 9) return pig;
    else if (!banana && !kiwi && length < 7) return duck;
    else                                     return dragon;
    \end{Verbatim}
    \item When writing a \texttt{do-while} function, put the \texttt{while} on the same line as the closing brace:
    \begin{Verbatim}[fontsize=\small]
    do
    {
        blabla;
    } while (bloebloe);
    \end{Verbatim}
    \item Indent \texttt{switch} statements like this:
    \begin{Verbatim}[fontsize=\small]
    switch (giraffeCount)
    {
        case 1: text = "one giraffe";  break;
        case 2: text = "two giraffes"; break;
        case 3:
            // If it's more than one line of code
            doStuffOnSeveralLines;
            text = "three giraffes";
            break;
        case 4:
        {
            // If you need more structure for readability, add curly braces
            int x = getComplexThing();
            text = "quadruple giraffe";
            break;
        }
    }
    \end{Verbatim}
    \item Function parameters have the exact same names in the \texttt{.h} and the \texttt{.cpp} files.
    \item Precompiler instructions (everything starting with \texttt{\#}) should be kept to a minimum, except of course for \texttt{\#include} and \texttt{\#pragma once}.
    \item Do not write macros.
    \item Variables inside functions are declared at the point where they are needed, not all of them at the start of the function.
    \item In constructors, prefer using initializer lists above setting variables in the body of the constructor. Each initialization in the initializer list gets its own line (see the example for exact formatting). Make sure variables in the initializer lists are in the same order as in the class definition in the \texttt{.h}-file.
    \item Do not use RTTI (so do not use \texttt{dynamic\_cast}). RTTI costs a little bit of performance, but more important is that RTTI is almost always an indication of bad object-oriented design.
    \item Use \texttt{reinterpret\_cast} and \texttt{const\_cast} only when absolutely necessary.
    \item Do not commit code that does not compile without errors or warnings (and do not disable warnings/errors either).
    \item Do not commit code that breaks existing functionality.
    \item No global variables.
    \item Always use namespaces explicitly. Do not put things like \texttt{using namespace std;} in code.
    \item Never ever even think of using \texttt{goto} statements. We have a thought detector and will electrocute you if you do.
    \item Don't use the comma operator, like in \texttt{if (i += 7, i < 10)}.
    \item Don't use \texttt{Unions}.
    \item Don't use function pointers, except where other libraries (like STL sort) require it.
    \item Only use the ternary operator (like for example \texttt{print(i > 5 ? “big” : “small”);}) in extremely simple cases. Never nest the ternary operator.
    \item When exposing a counter for an artist or designer it starts at 0, just like it does for normal arrays in code. Some old tools might still start at 1 but anything newly developed for artists starts at 0.
    \item Use \texttt{NULL} instead of 0 for pointers.
    \item When checking whether a pointer exists, make this explicit. So use \texttt{if (myPointer != NULL)} instead of \texttt{if (myPointer)}.
    \item Use RAII (Resource Acquisition Is Initialization) where possible. So instead of creating a separate function \texttt{initialise()}, fully initialize the class in its constructor so that it is never in an incomplete state.
    \item Write the constructor and destructor together: write the corresponding delete for every \texttt{new} immediately, so you do not forget to do it later on.
    \item Class definitions begin with all the functions and then all the variables. The order is \texttt{public/protected/private}. This might mean you have to use the keywords \texttt{public}, \texttt{protected}, \texttt{private} several times (first for functions, then for variables).
    \item A class starts with its constructors, followed by its destructor. If these are private or protected, then they should still be at the top.
    \item When something has two options, but is not clearly true/false, then prefer using an \texttt{enum class} instead of a boolean. For example, for direction don't use \texttt{bool isRight}. Instead, use \texttt{enum Direction} with the values \texttt{Left} and \texttt{Right}.
    \item Make sure all code is framerate-independent.
    \item Make the intended structure explicit and clear. For example, avoid doing things like this:
    \begin{Verbatim}[fontsize=\small]
    if (yellow)
    {
        return banana;
    }
    return kiwi;
    \end{Verbatim}
    What was really intended here, was that depending on \texttt{yellow}, either \texttt{banana} or \texttt{kiwi} is returned. It is more readable to make this explicit like this:
    \begin{Verbatim}[fontsize=\small]
    if (yellow)
    {
        return banana;
    }
    else
    {
        return kiwi;
    }
    \end{Verbatim}
    \item Use \texttt{nullptr} instead of \texttt{NULL}.
    \item We only use \texttt{auto} for things like complex iterator types.
    \item Use range-based \texttt{for} where applicable, e.g., \texttt{for (const Banana\& ultimateFruit : myList)}. Note that it's important to type that reference when not working with pointers, since otherwise the \texttt{Banana} would be copied in this case.
    \item Use \texttt{override} and \texttt{final} wherever applicable.
    \item If a function is virtual, the \texttt{virtual} keyword should always be added to it, so not only in the parent class, but also in each version of the function in the child classes.
    \item Don't use rvalue references (\texttt{\&\&}) unless you really, really, really need them.
    \item Use \texttt{unique\_ptr} wherever possible. If an object is not owned, then it's stored as a normal pointer. Avoid creating complex types in the initializer list. Here's an example of how that works:
    \begin{Verbatim}[fontsize=\small]
    class FruitManager
    {
    public:
        FruitManager(Kiwi* notMyFruit):
            notMyFruit(notMyFruit)
        {
            bestFruitOwnedHere.reset(new Banana());
        }
    private:
        Kiwi* notMyFruit;
        std::unique_ptr<Banana> bestFruitOwnedHere;
    };
    \end{Verbatim}
    \item Lambdas that are too big for a single line are indented like this:
    \begin{Verbatim}[fontsize=\small]
    auto it = std::find_if(myVec.begin(),
                           myVec.end(),
                           [id = 42] (const Element& e)
                           { 
                               return e.id() == id;
                           });
    \end{Verbatim}
    \item Don't use \texttt{MyConstructor = delete}.
    \item Every function uses an \texttt{enum} return to determine if the function succeeded or failed. The \texttt{enum} can be extended with different options.
\end{itemize}

\end{document}
