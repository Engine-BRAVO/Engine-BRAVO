\documentclass{article}
\usepackage{enumitem}

\title{Code Style Guidelines}
\author{}
\date{}

\begin{document}

\maketitle

\section*{Code Style Guidelines}

\begin{enumerate}[left=0pt, align=left]

    \item All code and comments should be in British English, e.g., use \texttt{colour}, \texttt{centre}, \texttt{initialiser} instead of \texttt{color}, \texttt{center}, \texttt{initializer}.

    \item Use proper spacing:
    \begin{itemize}
        \item Place a space after every comma, e.g., \texttt{myFunction(5, 7, 8)}.
        \item Surround operators with spaces, e.g., \texttt{5 + 7} instead of \texttt{5+7}.
    \end{itemize}

    \item Tabs should be the size of four spaces, and stored as tabs, not spaces.

    \item Ensure functions and static variables appear in the same order in \texttt{.h} and \texttt{.cpp} files.

    \item Use \texttt{\#pragma once} instead of include guards.

    \item Keep functions short (preferably no more than 50 lines) and avoid creating large classes. Limit class size to 750 lines; split classes only if the resulting code is clean.

    \item Use forward declarations to minimize header file includes. Place \texttt{\#include} directives in \texttt{.cpp} files wherever possible.

    \item Organize includes and forward declarations in the following order:
    \begin{itemize}
        \item Forward declarations from our code (alphabetically).
        \item Includes from our code (alphabetically).
        \item Empty line.
        \item Library-specific forward declarations and includes (alphabetically).
    \end{itemize}
    The header file associated with a \texttt{.cpp} file should always be included at the top.

    \item Define static variables as class members instead of within functions.

    \item Follow \textbf{const correctness}; only methods that cannot be \texttt{const} should omit it.

    \item Use descriptive names for variables and functions; avoid abbreviations unless widely known. Capitalize class, struct, and enum type names; start variable and function names with a lowercase letter. Example:
    \begin{verbatim}
        class MyClass
        {
            void someFunction();
            int someVariable;
        };
    \end{verbatim}

    \item Prefix member variables with \texttt{m} (e.g., \texttt{mMember}), and function arguments with \texttt{a} (e.g., \texttt{aArgument}).

    \item Do not implement functions in \texttt{.h} files. For template functions that must remain in the header, use a separate \texttt{MyClassImplementation.h}.

    \item Start template typenames with \texttt{T} (e.g., \texttt{TString}).

    \item Define only one class per header file, except for nested classes.

    \item Use two empty lines between functions and empty lines within functions for readability.

    \item Add extensive comments to the code. Use \texttt{///} for comments above classes and functions for compatibility with Linux tools.

    \item Use consistent brace styles:
    \begin{verbatim}
        if (condition)
        {
            // code
        }
        else
        {
            // code
        }
    \end{verbatim}

    \item Format \texttt{do-while} and \texttt{switch} statements as shown in the example below:
    \begin{verbatim}
        do
        {
            // code
        } while (condition);

        switch (variable)
        {
            case 1: action; break;
            case 2:
                // multiple lines
                action;
                break;
            case 3: { scoped action; break; }
        }
    \end{verbatim}

    \item Use consistent naming for parameters in header and implementation files.

    \item Limit precompiler instructions and avoid macros.

    \item Declare variables where they are first needed within functions.

    \item Use initializer lists in constructors, aligning them with the order in class definitions.

    \item Avoid using \texttt{dynamic\_cast}, \texttt{reinterpret\_cast}, and \texttt{const\_cast} unless necessary.

    \item Only commit code that compiles without warnings or errors.

    \item Avoid global variables and do not use \texttt{using namespace std}.

    \item Avoid using \texttt{goto} statements or unions.

    \item Limit the use of the ternary operator to simple cases.

    \item For user-facing counters, start from zero.

    \item Use \texttt{nullptr} for pointers, and make pointer existence checks explicit with \texttt{if (pointer != nullptr)}.

    \item Use RAII (Resource Acquisition Is Initialization) to initialize classes fully in constructors.

    \item Structure class definitions with functions followed by variables, ordered as \texttt{public}, \texttt{protected}, then \texttt{private}.

    \item Prefer \texttt{enum class} over \texttt{bool} for two-option values.

    \item Ensure code is framerate-independent and explicit in structure.

    \item Use \texttt{nullptr} instead of \texttt{NULL}, range-based \texttt{for} loops where possible, and only use \texttt{auto} for complex types.

    \item Apply \texttt{override} and \texttt{final} as needed and mark all virtual functions in subclasses.

    \item Use \texttt{unique\_ptr} for ownership; pass raw pointers only when ownership is not implied.

    \item Avoid \texttt{MyConstructor = delete} and use exceptions for error handling.

    \item Use C++ libraries (e.g., \texttt{cmath}) over C libraries (e.g., \texttt{math.h}).

    \item Favor smart pointers over traditional pointers.
    
    \item All files must contain Doxygen comments.
        \subitem Every file must contain a file tag.
        \subitem Every class must contain a class tag.
        \subitem Every function must contain a function tag, with the exeception of setters and getters that are trivial. 
        \subitem Every function argument must be documented. 
        \subitem Every return value must be documented.
        \subitem Every exception must be documented.
 
\end{enumerate}

\end{document}
