- Physics are updated sequentially in the normal game loop. It is checked how much time has accumulated since the last cycle, and the physics are ticked the required amount of times to keep up with the accumulated time.
  Alternatives considered:
    - Updating physics using a deltatime. This is undesirable, because when the update frequency of the game drops, the physics will also slow down and become unreliable and prone to error.
    - Multithreading. Not done because rendering can then copy old data and new data in a single frame. Alternative would be copying the data to physics and copying it back when physics is finished, but that would mean copying too much data back and forth per cycle.

- The 'levelManager' (state machine responsible for transitioning between scenes) is never automatically updated, it is only updated when called by a behavior script.

- Physics object types (in Box2D):
  - Staticbody: objects which can be manually moved, but are not affected by gravity, mass et cetera
  - Kinematic body: has zero mass, and can be moved by applying forces to it. It is not affected by gravity.
  - Dynamic body: has mass, and is affected by gravity and other forces. It can be moved by applying forces to it.

Staticbody is used for walls, and objects which may move but only when explicitly told to do so.
Kinematic bodies are not used, because the gravity of a dynamic body can instead be manually set to zero.
Dynamic bodies are used for all objects which are affected by gravity and / or other forces.

- GameObjects should only contain an animator *or* a sprite, not both. The renderer first checks if there is an animator present, and if not, it checks for a sprite.

- Controller input is supported, but not analog values. This is because it does not integrate well with SDL inputs.