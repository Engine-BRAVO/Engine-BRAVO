\section{Input}
\label{sec:input}
\subsection{Supported Input types}
\begin{itemize}
    \item \textbf{Keyboard input:} the engine supports all keys on the keyboard.
    \item \textbf{Mouse input:} the engine supports left, right, and middle mouse buttons.
    \item \textbf{Controller input:} the engine supports controller input, with analog values for the joysticks.
\end{itemize}

\subsection{Input class}
The input class is a singleton, and at the start of each cycle, it updates the input states.
This is done by checking the state of the input devices, and updating the input states accordingly.
After the input states are updated, the input class is responsible for checking if a key is pressed, released, or held down.
These calculations are done at the start of every cycle, and then the input states can be read from anywhere in the engine, but hopefully only used from the behavior scripts.

\subsection{Input contexts}
An input context is a state the input system can be in where it only allows certain inputs to go through.
For example, when the player is in the main menu, the input context is set to the main menu context, and only the inputs which are relevant to the main menu are allowed to go through.
And then when the player is in the game, the input context is set to the game context, and only the inputs which are relevant to the game are allowed to go through.
These contexts and inputs that are allowed can be set by the game programmer, through an config file.