\section{Physics Update}
Physics are updated sequentially in the normal game loop.
The physics are updated at a fixed rate of 50 Hz. This is done by checking how much time has passed since the last physics update, and then updating the physics the required amount of times to keep up with the accumulated time.

\subsection{Alternatives Considered}
\begin{itemize}
    \item Updating physics using a deltatime. This is undesirable, because when the update frequency of the game drops, the physics will also slow down and become unreliable and prone to error.
    \item Multithreading. Not done because rendering can then copy old data and new data in a single frame. Alternative would be copying the data to physics and copying it back when physics is finished, but that would mean copying too much data back and forth per cycle.
\end{itemize}

\section{Physics Library}
Engine bravo makes use of the Box2D library to process any physics logic within the application. Box2D is chosen due to recommendations from teachers and since it is one of the only regularly updated ans accessible 2d physics libraries.

\section{Physics Object Types (in Box2D)}
\begin{itemize}
    \item \textbf{Staticbody}: objects which can be manually moved, but are not affected by gravity, mass, etc.
    \item \textbf{Kinematic body}: has zero mass, and can be moved by applying forces to it. It is not affected by gravity.
    \item \textbf{Dynamic body}: has mass, and is affected by gravity and other forces. It can be moved by applying forces to it.
\end{itemize}

\noindent
\textbf{Usage:}
\begin{itemize}
    \item \textbf{Staticbody} is used for walls, and objects which may move but only when explicitly told to do so.
    \item \textbf{Kinematic bodies} are not used, because the gravity of a dynamic body can instead be manually set to zero.
    \item \textbf{Dynamic bodies} are used for all objects which are affected by gravity and/or other forces.
\end{itemize}

\section{PhysicsEngine}{
  \begin{itemize}
      \item Steps are at 50 Hz
      \item substeps are configurable by the game programmer
      \item update method gets int of the amount of steps it need to set so less data gets copied to the world class
  \end{itemize}
 }