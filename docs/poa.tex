% This is a simple sample document.  For more complicated documents take a look in the exercise tab. Note that everything that comes after a % symbol is treated as comment and ignored when the code is compiled.

\documentclass{article} % \documentclass{} is the first command in any LaTeX code.  It is used to define what kind of document you are creating such as an article or a book, and begins the document preamble

\usepackage{amsmath} % \usepackage is a command that allows you to add functionality to your LaTeX code
\usepackage{booktabs} % For better looking tables
\usepackage{geometry} % To adjust margins
\usepackage{tabularx} % For tables that auto-adjust to the page width

\title{Plan of Attack} % Sets article title
\author{Sean Groenenboom \and Seger Sars \and Siem Vermeulen \and Angel Villanueva \and Ronan Vlak} % Sets authors name
\date{\today} % Sets date for date compiled
\newpage

% The preamble ends with the command \begin{document}
\begin{document} % All begin commands must be paired with an end command somewhere
    \maketitle % creates title using information in preamble (title, author, date)
    \newpage

    \tableofcontents % creates a table of contents
    \newpage

    \section{work division}
    sean: Quality
    siem: General planning
    Seger:Project
    Angel: project orginisation, risk analysis
    Ronan:Document inventory, communication
    \newpage

    \section{version control}
    \newpage
            
    \section{Project}
    The following optional user stories shall be implemented
    \begin{itemize}
        \item Multiplayer / internet connection
    \end{itemize}
    \subsection{problem analysis}
    \subsection{scope}
    \subsection{goal}
    \subsection{result}
    \newpage

    \section{General planning} % creates a section
    Per week vastleggen wat er gedaan moet worden (zie papierwerk docu op BS)?

    \begin{tabularx}{\textwidth}{|X|X|X|X|}
    \hline
    \textbf{Week number} & \textbf{Activity} & \textbf{Finished this week} \\ \hline
    1 & Begin working on plan of attack & First plan of attack draft \\ \hline
      & Begin researching various potential features & Research document draft (1 feature per member) \\ \hline
    2 & Conduct further research on potential features &  \\ \hline
      & Determine future tasks based off research &  \\ \hline
      & Begin work on first POC & \\ \hline
    3 & Should Have &  \\ \hline
    4 & Should Have &  \\ \hline
    5 & Should Have &  \\ \hline
    6 & Should Have &  \\ \hline
    7 & Should Have &  \\ \hline
    8 & Should Have &  \\ \hline
    9 & Should Have &  \\ \hline
    10 & Should Have &  \\ \hline
    11 & Should Have &  \\ \hline
    12 & Should Have &  \\ \hline
    12 & Should Have &  \\ \hline
    14 & Should Have &  \\ \hline
    15 & Should Have &  \\ \hline
    16 & Should Have &  \\ \hline
    17 & Should Have &  \\ \hline
    18 & Should Have &  \\ \hline
    19 & Should Have &  \\ \hline
    20 & Should Have &  \\ \hline
    \end{tabularx}
    \label{tab:requirements}
    \newpage

    \section{Document inventory}
    Welke docus zijn er en wat staat daarin?
    \newpage

    \section{Communication}
    Hoe wordt er onderling gecommuniceerd (Whatsapp)
    Hoe vaak komt de groep samen? Op vaste momenten?
    \newpage

    \section{Quality}
    Vermeld overal bij welke tools worden gebruikt (en waarom)
    \subsection{Documentation}
    The following points will be applied to ensure the documentation’s quality:
    \begin{itemize}
        \item All documents are written in English.
        \item All documents are written in the present perfect
        \item All diagrams are in the UML style.
        \item Headers in the research document do not contain questions
        \item No choices are made in the research document
    \end{itemize}
    Wordt markdown gebruikt voor alle documenten (heeft goede compatibiliteit met version control)? Of zijn er betere alternatieven?
    \subsection{Technical}
    \begin{itemize}
        \item Code is written in english.
        \item Code is written in c++23.
        \item Doxygen is used for code documentation, except for the setters and getters
        \item Codestandard is written in ./codestandard.tex.
        \item Text editor of choice can be used.
        \item Cmake is used to build the code.
        \item Clangformat is used for code formatting.
        \item cpp check is used for code checking.
        \item valgrind is used for memory leak checking.
        \item gtest testing framework is used for testing.
        \item Unit test are written for every function except for setters without validation and getters.
        \item Third party libraries must be discussed with the whole team.
        \item One person must check git commits before merging.
        \item Only code which can be compiled may be commited to git.
    \end{itemize}

    Wat is het beleid voor versiemanagement? (welke tool(s), hoe / wanneer worden branches gemaakt, PRs reviewen etc.)
    Hoe wordt er getest? Unit tests, integratietest etc. Test framework voor gebruiken? Statische code check?
    \subsection{tools}
    \begin{itemize}
        \item Git is used for version management.
        \item Github is used to store the git repo.
        \item latex is used as default text editor. This is used so git can be used as version conrol.
        \item plantuml is used for UML diagrams. This is used so git can be used as version conrol.
        
    \end{itemize}
    \newpage

    \section{project orginisation}
    
    \subsection{Parties}
    
    \subsubsection{Customer and school supervisor}
    
    Bob van der Putten is the main customer of the project being the customer which the engine is built for to be used for developing games. Besides this, Bob is the school supervisor of the project and functions as a 
    
    \subsubsection{Project leader}
    
    Siem Vermeulen is the project leader and manages the planning of meetings and sending weekly progress the emails to the customer.
    
    \subsubsection{Developers}
    
    The software developers of the project are Sean Groenenboom, Siem Vermeulen, Angel Villanueva, and Ronan vlak. Developers have the responsibility of defining the project, researching solutions and developing and delivering the final product to the customer.
    
    
    \newpage

    \section{Risk analysis}
   
   	\subsection{Risks}
    
    \subsubsection{Technical}
    
    \begin{center}
    	\begin{tabular}{|c | c | c | c | c |} 
    		\hline
    		Nr. & Risk description & Impact (A) & Probability (B) & Priority (A*B) \\ [0.5ex] 
    		\hline\hline
    		1.1 & PC/laptop broken or damaged & 1 & 5  & 5 \\ 
    		\hline
    		1.2 & Library contains unknown bugs & 2 & 6 & 12 \\
    		\hline
    		1.3 & Library cannot integrate with other libraries & 2 & 6 & 12 \\
    		\hline
    		1.4 & OS compatibility & 5 & 5 & 25 \\
    		\hline
    		1.5 & IDE compatibility & 2 & 4 & 8 \\
    		\hline
    	\end{tabular}
    \end{center}

	\subsubsection{Project management}

	\begin{center}
		\begin{tabular}{|c | c | c | c | c |} 
			\hline
			Nr. & Risk description & Impact (A) & Probability (B) & Priority (A*B) \\ [0.5ex] 
			\hline\hline
			2.1 & Developer is sick and cannot work & 2 & 3  & 6 \\ 
			\hline
			2.2 & Project fails to meet planned deadlines & 2 & 4 & 8 \\
			\hline
			2.3 & Stakeholders/developers not available for meetings & 2 & 2 & 4 \\
			\hline
			2.4 & Lack/unclear communication within the project & 2 & 4 & 8 \\
			\hline
		\end{tabular}
	\end{center}
	
	\subsection{Measures}
	
	\subsubsection{Technical}
	
	\begin{center}
		\begin{tabular}{| p{0.05\linewidth} | p{0.4\linewidth} | p{0.4\linewidth} |}
			\hline
			Nr. & Consequence & Mitigation \\ [0.5ex] 
			\hline\hline
			1.1 & Broken equipment may cause a loss of progress or halt the progress for the affected developer &  A loss of progress is mitigated by consistently uploading code to Github \\
			\hline
			1.2 & May cause a loss of progress if discovered late into the development cycle & Research into libraries and creating POCs for the needed use cases can spot bugs in advance and provide alternatives if a problem arises later in the development cycle \\
			\hline
			1.2 & May cause a loss of progress if discovered late into the development cycle & Research into libraries and creating POCs for the needed use cases can spot bugs in advance and provide alternatives if a problem arises later in the development cycle \\
			\hline
			1.4 & Developer won't be able to work/test due to not being able to build/run the project & All developers make use of the same operating system \\
			\hline
			1.5 & Developer won't be able to work/test due to not being able to build/run the project  & Testing out developer preferred IDEs before starting project development \\ [1ex] 
			\hline
		\end{tabular}
	\end{center}
	
	\subsubsection{Project management}

	\begin{center}
		\begin{tabular}{| p{0.05\linewidth} | p{0.4\linewidth} | p{0.4\linewidth} |}
			\hline
			Nr. & Consequence & Mitigation \\ [0.5ex] 
			\hline\hline
			2.1 & Parts of the project may be delayed or remain unfinished & No mitigation \\
			\hline
			2.2 & Poor planning may result in a lack of time for certain aspects of the project and may lead to cuts in the project or an unfinished end result & Hosting weekly stand ups to discuss progress and providing buffers for potential delays \\
			\hline
			2.3 & May cause a delay in progress & Planning meetings a week or more in advance \\
			\hline
			2.4 & Unclear or insufficient communication may cause a lack of clarity regarding project progress & Hosting weekly stand ups and establishing robust communication channels between the team \\
			\hline
		\end{tabular}
	\end{center}
    
    \newpage

\end{document} % This is the end of the document


