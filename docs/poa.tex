% This is a simple sample document.  For more complicated documents take a look in the exercise tab. Note that everything that comes after a % symbol is treated as comment and ignored when the code is compiled.

\documentclass{article} % \documentclass{} is the first command in any LaTeX code.  It is used to define what kind of document you are creating such as an article or a book, and begins the document preamble

\usepackage{amsmath} % \usepackage is a command that allows you to add functionality to your LaTeX code
\usepackage{booktabs} % For better looking tables
\usepackage{geometry} % To adjust margins
\usepackage{tabularx} % For tables that auto-adjust to the page width
\usepackage{longtable} % For tables across multiple pages

\title{Plan of Attack} % Sets article title
\author{Sean Groenenboom \and Seger Sars \and Siem Vermeulen \and Angel Villanueva \and Ronan Vlak} % Sets authors name
\date{\today} % Sets date for date compiled
\newpage

% The preamble ends with the command \begin{document}
\begin{document} % All begin commands must be paired with an end command somewhere
    \maketitle % creates title using information in preamble (title, author, date)
    \newpage

    \tableofcontents % creates a table of contents
    \newpage

    \section{work division}
    sean: Quality
    siem: General planning
    Seger:Project
    Angel: project orginisation, risk analysis
    Ronan:Document inventory, communication
    \newpage

    \section{version control}
    All information in this document is subject to change. \\
    \begin{tabularx}{\textwidth}{|X|X|X|}
        \hline
        \textbf{Version number} & \textbf{Date} & \textbf{Change} \\ \hline
        V0.1 & 02-09-2024 & Initial draft \\ \hline
        V0.2 & 04-09-2024 & Created general planning and version control \\ \hline
        V0.3 & 06-09-2024 & Finished the initial version of risk analysis and project organisation \\ \hline
        \end{tabularx}
    \newpage
            
    \section{Project}
    % The following optional user stories shall be implemented
    % \begin{itemize}
    %     \item Multiplayer / internet connection
    % \end{itemize}
    \subsection{Problem analysis}
    CodedFun Games is a company that wants to scale its business, therefore
    the owner wants a custom-made game engine which can be maintained and expanded by a developer working 1 day a week.
    The engine is looking for a custom game engine that is easily maintainable, extendable and user-friendly, while also being well documented.
    The client has previous experience with Unity and wants the game engine to function and be accessible in a similar structure and API.
    \newline \newline
    Besides the game engine the client needs a simple validation game, so the client can test every feature and see how they are meant to be used.

    \subsection{Goal}
    The objective of this project is to create a game engine that the client can use for creating games at his company.
    The engine is designed to satisfy all requirements as stated in the requirements document enabling the client to make games with a custom engine without learning an entirely new game engine structure.

    \subsection{Scope}
    The game engine will be made to function on computers running a Linux operating system, because every developer uses a Linux operating system.
    \newline\newline
    The following is a list of items which will not be focused on while working on this project:  
    \begin{itemize}
        \item Graphical user interface for making a game.
        \item Direct compatibility with scripts from other game engines
        \item Advanced AI systems 
        \item Cross-platform compatibility
        \item High-End Graphics features
        \item In-Depth animation systems
    \end{itemize}

    \subsection{Result}
    The result of this project is a user-friendly game engine that follows the API structure of the Unity game engine. 
    The engine is easily understood, maintained and extendable by an engine engineer working one day a week.
    \newline\newline
    Besides the game engine a simple validation game is created to show all the functions of the engine.
    \newpage

    \section{General planning} % creates a section
    Deadlines in \textbf{bold} are mandated by school, all other deadlines are set by the team (and therefore theoretically flexible).
    Unless a specific day is specified, all deadlines are on friday, 17:00.
    \begin{longtable}{|l|p{0.4\textwidth}|p{0.4\textwidth}|}
    \hline
    \textbf{Week number} & \textbf{Activity} & \textbf{Deadlines} \\ \hline
    \endfirsthead

    \hline
    \textbf{Week number} & \textbf{Activity} & \textbf{Deadlines} \\ \hline
    \endhead

    \hline \multicolumn{3}{r}{\textit{Continued on the next page}} \\ \hline
    \endfoot

    \hline
    \endlastfoot

    1 & Begin working on plan of attack & PoA V1 \\ \hline
      & Research potential features & \\ \hline
    2 & Research potential features & Research draft (1st feature per member) \\ \hline
      & Determine future tasks based off research &  \\ \hline
      & Work on POCs & \\ \hline
    3 & Update PoA according to feedback &  \textbf{PoA (sunday)} \\ \hline
      & Work on POCs & Finish first POCs \\ \hline
      & Research potential features & Research draft (2nd feature per member) \\ \hline
    4 & Research potential features & Research V1 \\ \hline
      & Start drafting architecture + design &  \\ \hline
      & Work on POCs & Finish second POCs \\ \hline
    5 & Write and implement research feedback & Research feedack implemented \\ \hline
     & Work on POCs &  \\ \hline
      & Update architecture + design &  \\ \hline
    6 & Update architecture + design & Architecture V1 finished \\ \hline
      & Create demos from POCs & \\ \hline
      & Set up automated testing systems & Automated testing systems set up \\ \hline
    7 & Present POCs / demos to class &  \\ \hline
      & Write test plan & \\ \hline
      & Write design & \\ \hline
    8 & Update PoA &  \\ \hline
      & Write design & Design V1 \\ \hline
    9 & Write engine implementation &  \\ \hline
    10 & Write engine implementation &  \\ \hline
       & Review all work first period & \textbf{All work first period} \\ \hline
    11 & Write test plan & Test plan finished \\ \hline
       & Write engine implementation & \\ \hline
    12 & Write engine implementation &  \\ \hline
       & Review and update design & \\ \hline
       & Validation app design & \\ \hline
    13 & Write engine implementation & \\ \hline
       & Write test plan & \textbf{Present testplan} \\ \hline
       & Validation app design & Validation app design \\ \hline
    14 & Write engine implementation &  \\ \hline
       & Write validation app & \\ \hline
    15 & Write engine implementation & Engine V1 (all test cases passed) \\ \hline
       & Write validation app & \\ \hline
    16 & Optimize / refine engine &  \\ \hline
       & Write validation app & Finish vlaidation app \\ \hline
    17 & Optimize / refine engine & Engine finalized \\ \hline
       & & Showcase engine and validation app \\ \hline
    18 &  &  \\ \hline
    19 &  &  \\ \hline
    20 &  &  \\ \hline
    \end{longtable}
    \newpage

    \section{Document inventory}
    \begin{itemize}
        \item \textbf{Plan of Attack}
        \\
        A document containing information about the project's operational procedures and planning.
        \item \textbf{Requirements}
        \\
        A document containing information about the project requirements in a MoSCoW format
        \item \textbf{Research}
        \\
        A document containing research conducted for the project.
        \item \textbf {Design}
        \\
        A document outlining the design choices made for the project.
        \item \textbf {Test plan}
        \\
        A document providing information about the tests that are conducted.
        \item \textbf {Test results}
        \\
        A document containing the outcomes of tests that are conducted
    \end{itemize}

    \newpage

    \section{Communication}
    Communication between team members is done using whatsapp.
    \\
    Communication between the team and the Bob van der Putten 
    will be done using email, Teams or in person during the scheduled project meetings.
    \\\\
    Collaboration with code and documents is done using GitHub.
    When a team member wants to merge code, the code is reviewed by at least one other team member.
    \\\\
    Every wednesday at 9:15 the team is getting together to do a standup-meeting
     and work on the project.
    \\\\
    The team is working with a yellow and red card system.
    A yellow card means a warning. If two yellow cards are received by a teammember,
    they will be disqualified from participating in the project.
    The team is unable to issue a yellow card directly, but a recommendation to issue a card
     will be given to Bob van der Putten.
    \\
    A recommendation for a yellow card can only be issued when the following procedures have been followed:
    \begin{enumerate}
        \item A teammember is underperforming. This is the case when these conditions are met:
        \begin{itemize}
            \item A teammember has worked fewer hours than expected on the project.
            \item The quality of the submitted work is too low.
            \item A teammember has not communicated about issues they are facing within the project on time.
        \end{itemize}
        \item The team will make an arrangement with the underperforming teammember about the work that should be submitted.
              An email about this arrangement will also be sent to Bob van der Putten
        \item After one week, the arrangement is not met by the underperforming teammember.
    \end{enumerate}
    In case these procedures have all taken place, a recommendation for a yellow card will be given.


    \newpage

    \section{Quality}
    Vermeld overal bij welke tools worden gebruikt (en waarom)
    \subsection{Documentation}
    The following points will be applied to ensure the documentation’s quality:
    \begin{itemize}
        \item All documents are written in English.
        \item All documents are written in the present perfect.
        \item All diagrams are in the UML style.
        \item Headers in the research document do not contain questions. The research question is literally stated in the body of the research.
        \item No architecture or design choices are made in the research document.
        \item Every design choice in the corresponding document is explicitly motivated.
    \end{itemize}
    \subsection{Technical}
    \begin{itemize}
        \item Code is written in English.
        \item Code is written in C++23.
        \item Doxygen is used for code documentation. All classes, methods, namespaces and files are documented, with the exception of setters without validation and getters.
        \item All code is written in accordance with the codestandard document.
        \item Text editor of choice can be used.
        \item Cmake is used to build the code.
        \item Clangformat is used for code formatting.
        \item cpp check is used for static code checking.
        \item valgrind is used for memory leak checking.
        \item gtest testing framework is used for testing.
        \item Unit test are written for every function except for setters without validation and getters.
        \item Usage of third party libraries must be discussed with the whole team.
        \item At least one team member must approve pull requests before merging.
        \item Only code which can be compiled may be commited to git.
    \end{itemize}

    Wat is het beleid voor versiemanagement? (welke tool(s), hoe / wanneer worden branches gemaakt, PRs reviewen etc.)
    Hoe wordt er getest? Unit tests, integratietest etc. Test framework voor gebruiken? Statische code check?
    \subsection{tools}
    \begin{itemize}
        \item Git is used for version management.
        \item Github is used to store the git repo.
        \item latex is used as default text editor. This is used so git can be used as version conrol.
        \item plantuml is used for UML diagrams. This is used so git can be used as version conrol.
        
    \end{itemize}
    \newpage

    \section{project orginisation}
    
    \subsection{Parties}
    The project contains different parties consisting of developers and a customer. The following section provides information about the different parties, people involved within the parties and what their task is within the project.
    \subsubsection{Customer and school supervisor}
    
    Bob van der Putten is the main customer of the project and provides general demands and requirements for the engine. Besides the role of customer, Bob functions as the school supervisor and gives lectures regarding the general project and its factors. Finally, Bob  serves as the main talking point when it comes to discussing project progress.
    
   	\subsubsection{Developers}
    
    The software developers of the project are Sean Groenenboom, Siem Vermeulen, Angel Villanueva, and Ronan vlak. Developers have the responsibility of defining the project, researching solutions and developing and delivering the final product to the customer.
    
    
    \subsubsection{Project leader}
    
    Siem Vermeuelen has the role of project leader and oversees the project from within the developer team. The project leader is responsible for planning meetings, handing in documents and sending the weekly progress email to Bob van der Putten.
    
    \newpage

    \section{Risk analysis}
   
   	\subsection{Risks}
    
    \subsubsection{Technical}
    
    \begin{center}
    	\begin{tabular}{|c | c | c | c | c |} 
    		\hline
    		Nr. & Risk description & Impact (A) & Probability (B) & Priority (A*B) \\ [0.5ex] 
    		\hline\hline
    		1.1 & PC/laptop broken or damaged & 1 & 5  & 5 \\ 
    		\hline
    		1.2 & Library contains unknown bugs & 2 & 6 & 12 \\
    		\hline
    		1.3 & Library cannot integrate with other libraries & 2 & 6 & 12 \\
    		\hline
    		1.4 & OS compatibility & 5 & 5 & 25 \\
    		\hline
    		1.5 & IDE compatibility & 2 & 4 & 8 \\
    		\hline
    	\end{tabular}
    \end{center}

	\subsubsection{Project management}

	\begin{center}
		\begin{tabular}{|c | c | c | c | c |} 
			\hline
			Nr. & Risk description & Impact (A) & Probability (B) & Priority (A*B) \\ [0.5ex] 
			\hline\hline
			2.1 & Developer is sick and cannot work & 2 & 3  & 6 \\ 
			\hline
			2.2 & Project fails to meet planned deadlines & 2 & 4 & 8 \\
			\hline
			2.3 & Stakeholders/developers not available for meetings & 2 & 2 & 4 \\
			\hline
			2.4 & Lack/unclear communication within the project & 2 & 4 & 8 \\
			\hline
		\end{tabular}
	\end{center}
	
	\subsection{Measures}
	
	\subsubsection{Technical}
	
	\begin{center}
		\begin{tabular}{| p{0.05\linewidth} | p{0.4\linewidth} | p{0.4\linewidth} |}
			\hline
			Nr. & Consequence & Mitigation \\ [0.5ex] 
			\hline\hline
			1.1 & Broken equipment may cause a loss of progress or halt the progress for the affected developer &  A loss of progress is mitigated by consistently uploading code to Github \\
			\hline
			1.2 & May cause a loss of progress if discovered late into the development cycle & Research into libraries and creating POCs for the needed use cases can spot bugs in advance and provide alternatives if a problem arises later in the development cycle \\
			\hline
			1.2 & May cause a loss of progress if discovered late into the development cycle & Research into libraries and creating POCs for the needed use cases can spot bugs in advance and provide alternatives if a problem arises later in the development cycle \\
			\hline
			1.4 & Developer won't be able to work/test due to not being able to build/run the project & All developers make use of the same operating system \\
			\hline
			1.5 & Developer won't be able to work/test due to not being able to build/run the project  & Testing out developer preferred IDEs before starting project development \\ [1ex] 
			\hline
		\end{tabular}
	\end{center}
	
	\subsubsection{Project management}

	\begin{center}
		\begin{tabular}{| p{0.05\linewidth} | p{0.4\linewidth} | p{0.4\linewidth} |}
			\hline
			Nr. & Consequence & Mitigation \\ [0.5ex] 
			\hline\hline
			2.1 & Parts of the project may be delayed or remain unfinished & No mitigation \\
			\hline
			2.2 & Poor planning may result in a lack of time for certain aspects of the project and may lead to cuts in the project or an unfinished end result & Hosting weekly stand ups to discuss progress and providing buffers for potential delays \\
			\hline
			2.3 & May cause a delay in progress & Planning meetings a week or more in advance \\
			\hline
			2.4 & Unclear or insufficient communication may cause a lack of clarity regarding project progress & Hosting weekly stand ups and establishing robust communication channels between the team \\
			\hline
		\end{tabular}
	\end{center}
    
    \newpage

\end{document} % This is the end of the document


