% This is a simple sample document.  For more complicated documents take a look in the exercise tab. Note that everything that comes after a % symbol is treated as comment and ignored when the code is compiled.

\documentclass{article} % \documentclass{} is the first command in any LaTeX code.  It is used to define what kind of document you are creating such as an article or a book, and begins the document preamble

\usepackage{amsmath} % \usepackage is a command that allows you to add functionality to your LaTeX code

\title{Plan of Attack} % Sets article title
\author{Sean Groenenboom \and Seger Sars \and Siem Vermeulen \and Angel Villanueva \and Ronan Vlak} % Sets authors name
\date{\today} % Sets date for date compiled
\newpage

% The preamble ends with the command \begin{document}
\begin{document} % All begin commands must be paired with an end command somewhere
    \maketitle % creates title using information in preamble (title, author, date)
    \newpage

    \tableofcontents % creates a table of contents
    \newpage

    \section{work division}
    sean: Quality
    siem: General planning
    Seger:Project
    Angel: project orginisation, risk analysis
    Ronan:Document inventory, communication
    \newpage

    \section{version control}
    \newpage
            
    \section{Project}
    % The following optional user stories shall be implemented
    % \begin{itemize}
    %     \item Multiplayer / internet connection
    % \end{itemize}
    \subsection{Problem analysis}
    CodedFun Games is a company that wants to scale its business, therefore
    the owner wants a custom-made game engine which can be maintained and expanded by a developer working 1 day a week.
    The engine is looking for a custom game engine that is easily maintainable, extendable and user-friendly, while also being well documented.
    The client has previous experience with Unity and wants the game engine to function and be accessible in a similar structure and API.
    \newline \newline
    Besides the game engine the client needs a simple validation game, so the client can test every feature and see how they are meant to be used.

    \subsection{Goal}
    The objective of this project is to create a game engine that the client can use for creating games at his company.
    The engine is designed to satisfy all requirements as stated in the requirements document enabling the client to make games with a custom engine without learning an entirely new game engine structure.

    \subsection{Scope}
    The game engine will be made to function on computers running a Linux operating system, because every developer uses a Linux operating system.
    \newline\newline
    The following is a list of items which will not be focused on while working on this project:  
    \begin{itemize}
        \item Graphical user interface for making a game.
        \item Direct compatibility with scripts from other game engines
        \item Advanced AI systems 
        \item Cross-platform compatibility
        \item High-End Graphics features
        \item In-Depth animation systems
    \end{itemize}

    \subsection{Result}
    The result of this project is a user-friendly game engine that follows the API structure of the Unity game engine. 
    The engine is easily understood, maintained and extendable by an engine engineer working one day a week.
    \newline\newline
    Besides the game engine a simple validation game is created to show all the functions of the engine.
    \newpage

    \section{General planning} % creates a section
    Per week vastleggen wat er gedaan moet worden (zie papierwerk docu op BS)? 
    \newpage

    \section{Document inventory}
    Welke docus zijn er en wat staat daarin?
    \newpage

    \section{Communication}
    Hoe wordt er onderling gecommuniceerd (Whatsapp)
    Hoe vaak komt de groep samen? Op vaste momenten?
    \newpage

    \section{Quality}
    Vermeld overal bij welke tools worden gebruikt (en waarom)
    \subsection{Documentation}
    Wordt markdown gebruikt voor alle documenten (heeft goede compatibiliteit met version control)? Of zijn er betere alternatieven?
    Welke taal worden de documenten in geschreven?
    Wordt voor diagrammen UML aangehouden?
    Wat voor code comments worden er geschreven? (Doxygen?)
    \subsection{Technical}
    Welke taal worden variabele namen en comments etc. geschreven?
    Wordt er een code standaard aangehouden? Welke en waarom?
    Welke OS en IDE worden er gebruikt? Iedereen dezelfde? (C++ BS course adviseert CLion + CMake, vanwege cross platform mogelijkheden. Welke OS hebben Angel en Ronan?)
    Welke codestyle wordt er aangehouden? (automatische style tool?)
    Wat is het beleid voor versiemanagement? (welke tool(s), hoe / wanneer worden branches gemaakt, PRs reviewen etc.)
    Hoe wordt er getest? Unit tests, integratietest etc. Test framework voor gebruiken? Statische code check?
    \subsection{tools}
    \newpage

    \section{project orginisation}
    \newpage

    \section{Risk analysis}
    \newpage

\end{document} % This is the end of the document


