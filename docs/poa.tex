% This is a simple sample document.  For more complicated documents take a look in the exercise tab. Note that everything that comes after a % symbol is treated as comment and ignored when the code is compiled.

\documentclass{article} % \documentclass{} is the first command in any LaTeX code.  It is used to define what kind of document you are creating such as an article or a book, and begins the document preamble

\usepackage{amsmath} % \usepackage is a command that allows you to add functionality to your LaTeX code
\usepackage{booktabs} % For better looking tables
\usepackage{geometry} % To adjust margins
\usepackage{tabularx} % For tables that auto-adjust to the page width

\title{Plan of Attack} % Sets article title
\author{Sean Groenenboom \and Seger Sars \and Siem Vermeulen \and Angel Villanueva \and Ronan Vlak} % Sets authors name
\date{\today} % Sets date for date compiled
\newpage

% The preamble ends with the command \begin{document}
\begin{document} % All begin commands must be paired with an end command somewhere
    \maketitle % creates title using information in preamble (title, author, date)
    \newpage

    \tableofcontents % creates a table of contents
    \newpage

    \section{work division}
    sean: Quality
    siem: General planning
    Seger:Project
    Angel: project orginisation, risk analysis
    Ronan:Document inventory, communication
    \newpage

    \section{version control}
    \newpage
            
    \section{Project}
    The following optional user stories shall be implemented
    \begin{itemize}
        \item Multiplayer / internet connection
    \end{itemize}
    \subsection{problem analysis}
    \subsection{scope}
    \subsection{goal}
    \subsection{result}
    \newpage

    \section{General planning} % creates a section
    Per week vastleggen wat er gedaan moet worden (zie papierwerk docu op BS)?

    \begin{tabularx}{\textwidth}{|X|X|X|X|}
    \hline
    \textbf{Week number} & \textbf{Activity} & \textbf{Finished this week} \\ \hline
    1 & Begin working on plan of attack & First plan of attack draft \\ \hline
      & Begin researching various potential features & Research document draft (1 feature per member) \\ \hline
    2 & Conduct further research on potential features &  \\ \hline
      & Determine future tasks based off research &  \\ \hline
      & Begin work on first POC & \\ \hline
    3 & Should Have &  \\ \hline
    4 & Should Have &  \\ \hline
    5 & Should Have &  \\ \hline
    6 & Should Have &  \\ \hline
    7 & Should Have &  \\ \hline
    8 & Should Have &  \\ \hline
    9 & Should Have &  \\ \hline
    10 & Should Have &  \\ \hline
    11 & Should Have &  \\ \hline
    12 & Should Have &  \\ \hline
    12 & Should Have &  \\ \hline
    14 & Should Have &  \\ \hline
    15 & Should Have &  \\ \hline
    16 & Should Have &  \\ \hline
    17 & Should Have &  \\ \hline
    18 & Should Have &  \\ \hline
    19 & Should Have &  \\ \hline
    20 & Should Have &  \\ \hline
    \end{tabularx}
    \label{tab:requirements}
    \newpage

    \section{Document inventory}
    \begin{itemize}
        \item \textbf{Plan of Attack}
        \\
        A document containing information about the project's operational procedures and planning.
        \item \textbf{Requirements}
        \\
        A document containing information about the project requirements in a MoSCoW format
        \item \textbf{Research}
        \\
        A document containing research conducted for the project.
        \item \textbf {Design}
        \\
        A document outlining the design choices made for the project.
        \item \textbf {Test plan}
        \\
        A document providing information about the tests that are conducted.
        \item \textbf {Test results}
        \\
        A document containing the outcomes of tests that are conducted
    \end{itemize}

    \newpage

    \section{Communication}
    Communication between team members is done using whatsapp.
    \\
    Communication between the team and the Bob van der Putten 
    will be done using email, Teams or in person during the scheduled project meetings.
    \\\\
    Collaboration with code and documents is done using GitHub.
    When a team member wants to merge code, the code is reviewed by at least one other team member.
    \\\\
    Every wednesday at 9:15 the team is getting together to do a standup-meeting
     and work on the project.
    \\\\
    The team is working with a yellow and red card system.
    A yellow card means a warning. If two yellow cards are received by a teammember,
    they will be disqualified from participating in the project.
    The team is unable to issue a yellow card directly, but a recommendation to issue a card
     will be given to Bob van der Putten.
    \\
    A recommendation for a yellow card can only be issued when the following procedures have been followed:
    \begin{enumerate}
        \item A teammember is underperforming. This is the case when these conditions are met:
        \begin{itemize}
            \item A teammember has worked fewer hours than expected on the project.
            \item The quality of the submitted work is too low.
            \item A teammember has not communicated about issues they are facing within the project on time.
        \end{itemize}
        \item The team will make an arrangement with the underperforming teammember about the work that should be submitted.
              An email about this arrangement will also be sent to Bob van der Putten
        \item After one week, the arrangement is not met by the underperforming teammember.
    \end{enumerate}
    In case these procedures have all taken place, a recommendation for a yellow card will be given.


    \newpage

    \section{Quality}
    Vermeld overal bij welke tools worden gebruikt (en waarom)
    \subsection{Documentation}
    The following points will be applied to ensure the documentation’s quality:
    \begin{itemize}
        \item All documents are written in English.
        \item All documents are written in the present perfect
        \item All diagrams are in the UML style.
        \item Headers in the research document do not contain questions
        \item No choices are made in the research document
    \end{itemize}
    Wordt markdown gebruikt voor alle documenten (heeft goede compatibiliteit met version control)? Of zijn er betere alternatieven?
    \subsection{Technical}
    \begin{itemize}
        \item Code is written in english.
        \item Code is written in c++23.
        \item Doxygen is used for code documentation, except for the setters and getters
        \item Codestandard is written in ./codestandard.tex.
        \item Text editor of choice can be used.
        \item Cmake is used to build the code.
        \item Clangformat is used for code formatting.
        \item cpp check is used for code checking.
        \item valgrind is used for memory leak checking.
        \item gtest testing framework is used for testing.
        \item Unit test are written for every function except for setters without validation and getters.
        \item Third party libraries must be discussed with the whole team.
        \item One person must check git commits before merging.
        \item Only code which can be compiled may be commited to git.
    \end{itemize}

    Wat is het beleid voor versiemanagement? (welke tool(s), hoe / wanneer worden branches gemaakt, PRs reviewen etc.)
    Hoe wordt er getest? Unit tests, integratietest etc. Test framework voor gebruiken? Statische code check?
    \subsection{tools}
    \begin{itemize}
        \item Git is used for version management.
        \item Github is used to store the git repo.
        \item latex is used as default text editor. This is used so git can be used as version conrol.
        \item plantuml is used for UML diagrams. This is used so git can be used as version conrol.
        
    \end{itemize}
    \newpage

    \section{project orginisation}
    \newpage

    \section{Risk analysis}
    \newpage

\end{document} % This is the end of the document


