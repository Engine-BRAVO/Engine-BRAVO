% This is a simple sample document.  For more complicated documents take a look in the exercise tab. Note that everything that comes after a % symbol is treated as comment and ignored when the code is compiled.

\documentclass{article} % \documentclass{} is the first command in any LaTeX code.  It is used to define what kind of document you are creating such as an article or a book, and begins the document preamble

\usepackage{amsmath} % \usepackage is a command that allows you to add functionality to your LaTeX code
\usepackage{booktabs} % For better looking tables
\usepackage{geometry} % To adjust margins
\usepackage{tabularx} % For tables that auto-adjust to the page width

\title{Plan of Attack} % Sets article title
\author{Sean Groenenboom \and Seger Sars \and Siem Vermeulen \and Angel Villanueva \and Ronan Vlak} % Sets authors name
\date{\today} % Sets date for date compiled
\newpage

% The preamble ends with the command \begin{document}
\begin{document} % All begin commands must be paired with an end command somewhere
    \maketitle % creates title using information in preamble (title, author, date)
    \newpage

    \tableofcontents % creates a table of contents
    \newpage

    \section{work division}
    sean: Quality
    siem: General planning
    Seger:Project
    Angel: project orginisation, risk analysis
    Ronan:Document inventory, communication
    \newpage

    \section{version control}
    All information in this document is subject to change. \\
    \begin{tabularx}{\textwidth}{|X|X|X|}
        \hline
        \textbf{Version number} & \textbf{Date} & \textbf{Change} \\ \hline
        V0.1 & 02-09-2024 & Initial draft \\ \hline
        V0.2 & 04-09-2024 & Created general planning and version control \\ \hline
        \end{tabularx}
    \newpage
            
    \section{Project}
    The following optional user stories shall be implemented
    \begin{itemize}
        \item Multiplayer / internet connection
    \end{itemize}
    \subsection{problem analysis}
    \subsection{scope}
    \subsection{goal}
    \subsection{result}
    \newpage

    \section{General planning} % creates a section
    Deadlines in \textbf{bold} are mandated by school, all other deadlines are set by the team (and therefore theoretically flexible).
    Unless a specific day is specified, all deadlines are on friday, 17:00.
    \begin{tabularx}{\textwidth}{|X|X|X|X|}
    \hline
    \textbf{Week number} & \textbf{Activity} & \textbf{Deadlines} \\ \hline
    1 & Begin working on plan of attack & PoA V1 (friday) \\ \hline
      & Research potential features & Research draft (1st feature per member) \\ \hline
    2 & Research potential features & Research draft (2nd feature per member) \\ \hline
      & Determine future tasks based off research &  \\ \hline
      & Work on POCs & \\ \hline
    3 & Update PoA according to feedback &  \textbf{PoA (sunday)} \\ \hline
      & Work on POCs & Finish first POCs \\ \hline
      & Research potential features & \\ \hline
    4 & Research potential features & Research V1 \\ \hline
      & Start drafting architecture + design &  \\ \hline
      & Work on POCs & Finish second POCs \\ \hline
    5 & Write and implement research feedback & Research feedack implemented \\ \hline
     & Work on POCs &  \\ \hline
      & Update architecture + design &  \\ \hline
    6 & Update architecture + design & Architecture V1 finished \\ \hline
      & Create demos from POCs & \\ \hline
    7 & Present POCs / demos to class &  \\ \hline
    8 & Update PoA &  \\ \hline
    9 & Should Have &  \\ \hline
    10 & Should Have & \textbf{All work first period} \\ \hline
    11 & Should Have &  \\ \hline
    12 & Should Have &  \\ \hline
    12 & Should Have &  \\ \hline
    14 & Should Have &  \\ \hline
    15 & Should Have &  \\ \hline
    16 & Should Have &  \\ \hline
    17 & Should Have &  \\ \hline
    18 & Should Have &  \\ \hline
    19 & Should Have &  \\ \hline
    20 & Should Have &  \\ \hline
    \end{tabularx}
    \newpage

    \section{Document inventory}
    Welke docus zijn er en wat staat daarin?
    \newpage

    \section{Communication}
    Hoe wordt er onderling gecommuniceerd (Whatsapp)
    Hoe vaak komt de groep samen? Op vaste momenten?
    \newpage

    \section{Quality}
    Vermeld overal bij welke tools worden gebruikt (en waarom)
    \subsection{Documentation}
    The following points will be applied to ensure the documentation’s quality:
    \begin{itemize}
        \item All documents are written in English.
        \item All documents are written in the present perfect
        \item All diagrams are in the UML style.
        \item Headers in the research document do not contain questions
        \item No choices are made in the research document
    \end{itemize}
    Wordt markdown gebruikt voor alle documenten (heeft goede compatibiliteit met version control)? Of zijn er betere alternatieven?
    \subsection{Technical}
    \begin{itemize}
        \item Code is written in english.
        \item Code is written in c++23.
        \item Doxygen is used for code documentation, except for the setters and getters
        \item Codestandard is written in ./codestandard.tex.
        \item Text editor of choice can be used.
        \item Cmake is used to build the code.
        \item Clangformat is used for code formatting.
        \item cpp check is used for code checking.
        \item valgrind is used for memory leak checking.
        \item gtest testing framework is used for testing.
        \item Unit test are written for every function except for setters without validation and getters.
        \item Third party libraries must be discussed with the whole team.
        \item One person must check git commits before merging.
        \item Only code which can be compiled may be commited to git.
    \end{itemize}

    Wat is het beleid voor versiemanagement? (welke tool(s), hoe / wanneer worden branches gemaakt, PRs reviewen etc.)
    Hoe wordt er getest? Unit tests, integratietest etc. Test framework voor gebruiken? Statische code check?
    \subsection{tools}
    \begin{itemize}
        \item Git is used for version management.
        \item Github is used to store the git repo.
        \item latex is used as default text editor. This is used so git can be used as version conrol.
        \item plantuml is used for UML diagrams. This is used so git can be used as version conrol.
        
    \end{itemize}
    \newpage

    \section{project orginisation}
    \newpage

    \section{Risk analysis}
    \newpage

\end{document} % This is the end of the document


