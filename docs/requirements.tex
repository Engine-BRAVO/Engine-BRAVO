% This is a simple sample document.  For more complicated documents take a look in the exercise tab. Note that everything that comes after a % symbol is treated as comment and ignored when the code is compiled.

\documentclass{article} % \documentclass{} is the first command in any LaTeX code.  It is used to define what kind of document you are creating such as an article or a book, and begins the document preamble

\usepackage{amsmath} % \usepackage is a command that allows you to add functionality to your LaTeX code
\usepackage{booktabs} % For better looking tables
\usepackage{geometry} % To adjust margins
\usepackage{tabularx} % For tables that auto-adjust to the page width

\title{Requirements} % Sets article title
\author{Sean Groenenboom \and Seger Sars \and Siem Vermeulen \and Angel Villanueva \and Ronan Vlak} % Sets authors name
\date{\today} % Sets date for date compiled
\newpage

% The preamble ends with the command \begin{document}
\begin{document} % All begin commands must be paired with an end command somewhere
    \maketitle % creates title using information in preamble (title, author, date)
    \newpage

    \section{Requirements}
    \begin{table}[h!]
        \centering
        \begin{tabularx}{\textwidth}{|X|X|X|X|}
        \hline
        \textbf{Requirement Number} & \textbf{Requirement Type (MoSCoW)} & \textbf{Requirement Name} & \textbf{Definition of Done} \\ \hline
        M1 & Must Have & Levels support & The engine shall support multiple regions (levels) in which the game can be played. \\ \hline
        M2 & Must Have & Scenes support & The engine shall support multiple scenes (sections in the game which are significantly different from each other e.g. menu, gameplay, credits). \\ \hline
        M3 & Must Have & Keyboard support & The engine can use inputs from all keys of a 75\% size keyboard to use in the game. \textit{\textbf{Is 75\% reasonable? Or should we pick something else?}} \\ \hline
        M4 & Must Have & Mouse support & The engine can use the following inputs from a computer mouse: horizontal and vertical movement, left, right and middle mouse button and scroll wheel. \\ \hline
        S1 & Should Have & Online or LAN multiplayer support & \\ \hline
        S2 & Should Have & FPS can be fixed at a maximum & \\ \hline
        \end{tabularx}
        \caption{Requirements Table}
        \label{tab:requirements}
        \end{table}
    \newpage

\end{document} % This is the end of the document
