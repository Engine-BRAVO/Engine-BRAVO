\section{Levels (Ronan)}
\subsection{Levels in videogames}
A level in a videogame is a space available to a player during
the completion of an objective. Levels are designed to present a challenge,
objective, or puzzle that players must overcome to advance.
There are a few aspects that make up a level in a videogame:
\begin{itemize}
    \item Progression
    \item Design/Environment
    \item Objectives
    \item Difficulty
    \item Possible story advancement
\end{itemize}

\subsection{Level transitions}
Level transitions allow a player from level to another.
Level transitions can be one or two way.
A two way level transition can be used so a player can travel back and forth between connected levels.
One way level transitions prevent players from the traveling back to previous levels.
There a few common ways to manage the transition between levels:
\begin {itemize}
\item \textbf{Hard cut transition} \\
The game instantly switches from one level to another without any visuals.
An example that uses this type of transition is Pac-Man.
\item \textbf{Loading screen} \\
The game shows a loading screen with some type of progress bar to move to a new level.
This way is common in games that use large, resource-heavy levels.
\item \textbf{Teleportation} \\
Some type of portal or teleporter which transfers a player from level to level.
\item \textbf{Seamless transition} \\
A seamless transition gives a player the impression that they stay in the same location while switching between levels.
This is done by having two levels have an area in common, like a door or a hallway.
This type of transition is commonly used in rogue-like games, where the player progresses through some type of dungeon.
\end{itemize}

\subsection{Scenes}
In game development, scenes are different parts of a game, each showing a different state or setting.
A scene usually changes when the background or environment changes.
For example, the main menu, a map screen, and the gameplay itself are all separate scenes.
\\
Scenes are different from levels.
A level is a playable area where the player interacts, but a scene can also include non-playable parts like menus or pause screens.
Essentially, a scene represents a different state of the game, whether it’s gameplay or a menu.
Scenes are the basic building blocks used to organize and structure a game.


\subsection{Storing levels}
There are a few common ways to store levels in game engines:
\subsubsection {Scene files}
Existing game engines like Unity or Unreal engine use built-in systems to store levels in specific formats.
For example, Unity stores levels as .unity files and Unreal engine uses .umap files.
\\\\
\textbf{Advantages:}
\begin{itemize}
    \item Levels are easy to manage.
    \item Levels are visually editable within the existing game engines.
\end{itemize}

\noindent\textbf{Disadvantages:}
\begin{itemize}
    \item Levels are not portable across different engines.
    \item Levels cannot be edited outside the specific game engine itself.
\end{itemize}

\subsubsection{Text-based files}
Levels can be stored in text-based formats like JSON or XML.
The text describes the layout, assets and properties of a level.
\\\\
\textbf{Advantages:}
\begin{itemize}
    \item Level files are readable for humans.
    \item Levels can be edited outside of game engines.
\end{itemize}

\noindent\textbf{Disadvantages:}
\begin{itemize}
    \item Complex levels can become hard to manage.
    \item Parsing a level can be slow.
\end{itemize}

\subsubsection{Tilemaps}
2D games can use tilemaps to store levels. This way a level is made up of small, resuable tiles.
The tilemaps are stored as arrays where each value corresponds to a specific tile on a grid.
\\\\
\textbf{Advantages:}
\begin{itemize}
    \item Levels are stored efficiently for grid-based games.
    \item Levels are easy to modify.
\end{itemize}

\noindent\textbf{Disadvantages:}
\begin{itemize}
    \item This way of storing levels is limited to grid-based games.
    \item Tilemaps cannot be used for 3D levels.
\end{itemize}

\subsubsection{Binary files}
Level data can be stored in a binary format, which is more compact and efficient than text-based formats.
\\\\
\textbf{Advantages:}
\begin{itemize}
    \item Levels are stored in small files sizes.
\end{itemize}
\noindent\textbf{Disadvantages:}
\begin{itemize}
    \item Files are not readable for humans.
    \item Levels can be harder to debug or manually edit.
\end{itemize}

\subsubsection{Procedural generation}
Levels can be generated using a small set of numbers (Seeds) instead of storing a complete level.
This is done using algorithms to dynamically generate levels while the game is active.
\\\\
\textbf{Advantages:}
\begin{itemize}
    \item Generating levels this way requires minimal storage space.
    \item There are infinite variations of levels possible.
\end{itemize}
\noindent\textbf{Disadvantages:}
\begin{itemize}
    \item Editing specific level designs can be hard.
    \item Algorithms to generate the levels might be complex to create.
\end{itemize}

\subsection{2D third party level editors}
\subsubsection {Tiled}
Tiled is one of the most popular 2D level editors.
It is an open-source project with a large community.
The primary feature of Tiled is create and edit tile maps, but it also support free image placement
and level annotation. Tiled also support different object layers.
Levels created with Tiled can be exported in many different formats, including JSON and XML. The primary export format used by Tiled is a .tmx file, which uses XML.
Created levels that are stored in a .tmx format can be parsed in c++ using existing libraries, like tinytmx. (See POC\_Tiled2D)

\subsubsection {Ogmo}
Ogmo editor is an open-source 2D level editor.
It is focussed on fast and efficient level design.
Ogmo is primarily designed for grid-based maps.
Levels created with Ogmo are exported in a JSON format.

\subsubsection {LDtk}
LDtk is a newer open-source 2D level editor that uses tilemaps.
It is focussed primarily on ease of use.
LDtk's features include customizable layers, entities and enums.
Levels created with LDtk are exported in a JSON format.
