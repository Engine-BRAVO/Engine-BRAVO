\section{Physics}

Physics within games are a key component when it comes to creating realism between the interaction of objects and their physical properties such as velocity, density etc.

\subsection{The use of physics engines}

Within video games, physics engines make use of mathematical algorithms and models to realistically calculate the physical behaviour of objects within a game \cite{Ipacs_2023}. The data produced by these engines can then be used to animate objects and detect collision between objects. This method saves time and creates more realism when animating objects.

\subsection{Examples of physics engines}

\subsubsection{Box2D}

Box2D is currently one of the most popular 2D physics libraries on github and describes itself as a "2D rigid body simulation library for games" \cite{Catto_2024}.

\subsubsection{LiquidFun}

LiquidFun is an expansion on the previously mentioned Box2D (reference to previous section). It contains the original features of Box2D but expands on it by adding "particle based fluid and soft body simulation" \cite{Miles_2014}.

\subsubsection{Chipmunk2D}

Chipmunk2D aims to provide "a simple, lightweight, fast and portable 2D rigid body physics library" \cite{Slembcke_2023}.
